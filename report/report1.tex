\documentclass{repMobRob}


\author{Group F - Prop Heller \\ D. Baur, F. Fernandes Pinto, Q. Herzig, W. Ponsot, F. Reinhard and B. Rico Caldas}
\teacher{Julien Lecoeur}
\semester{Autumn 2015}
\cours{Mobile Robots}
\title{Miniproject}
\subtitle{Altitude control \\ \small{Project organization}}



\begin{document}
\thispagestyle{empty}
\maketitle

\section{Sensors}
\subsection{Sonar}
The sonar delivers a high resolution and low noise measurement compared to the other sensors (\SI{1}{\centi\meter} resolution with noise in the same order of magnitude).
But it's a measure of height (as opposed to a measure of altitude) and is not sufficent to control the altitude.
Furthermore, its sampling frequency (\SI{40}{\hertz}) is lower then the frequency of our \emph{contorl-loop}.

\subsection{Barometer}
The barometer is less precise than the sonar (typical \emph{RMS noise} of \SI{25}{\centi\meter} and a resolution of \SI{20}{\centi\meter}).
It returns an absolute altitude measurement, but it's drift is very limiting.
Also the changes of pressure induced by the environment (e.g. closing or opening a door) will distort the measurements.
The sampling frequency is the same as the sonar's when using the ultra high resolution mode (\SI{40}{\hertz}).

\subsection{Accelerometer}
The accelerometer isn't an altitude sensor, but provides us with information about the quadrotor movements.
Its bias is very important and the measurement can't just be integrated twice to yeild a position information without being fused with other sensors.
Its sampling frenquency is \SI{250}{\hertz} and thus the only one coming close to our \emph{control-loop} frequency. 

\section{Altitude determination}
Giving the fact that the sensors aren’t precise enough and cannot individually process the box insertion, we need to combine multiple sensors. 

\subsection{Complementary filter}
\subsubsection{Barometer \& sonar}
As these sensor's sampling frequency is slow, the quadrotor's velocity needs also to bo low to respect Shannon's criterion. This approach is certainly not good enough to success the different tests, as the raising time is limited. Additionally, the barometer has a lot of drift, which would yield to a complex altitude estimation.

\subsubsection{Accelerometer \& sonar}
By applying a \emph{low-pass} filter on the sonar and a \emph{high-pass} on the accelerometer integrated twice, we would theoretically have a good altitude estimation. This solution will potentially success the box test since the altitude's change will be noticed only by the sonar, while the accelerometer won't spot any change. Therefore, we can assume that the quadrotor altitude doesn't need to be updated.

\subsection{Kalman filter}
The bigger the state vector, the better the altitude estimation. However, the complexity increases and the model becomes tricky to characterize. 

\subsection{Solution kept}
Kalman filter yields to the best results but is too binding in terms of complexity to characterize the model properly. This is why we chose the accelerometer \& sonar complementary filter solution. 

\section{Altitude control}
We plan to use two cascaded \emph{PIDs} to control the quadrotor's velocity and position. Like that we can split the problem by tuning two controllers separately.

\section{Planning}
\begin{table}[H]
	\centering
	\caption{Planning}
	\label{tab:planning}
	\begin{tabular}{l|l|l}
        \hline
        \textbf{Task} & Timespan & Description \\
        \hline
		\textbf{Altitude estimation} & 23.11 - 29.11 (2 students) & Design and tuning of the complementary filter \\
		\textbf{Altitude control}    & 30.11 - 06.12 (2 students) & Design and tuning of the 2 cascaded PID       \\
		\textbf{Tests}               & 07.12 - 13.12 (2 students) & Test flights, final tunings                   \\
        \hline
	\end{tabular}
\end{table}

\end{document}
