\documentclass{repMobRob}


\author{Group F - Prop Heller \\ D. Baur, F. Fernandes Pinto, Q. Herzig, W. Ponsot, F. Reinhard and B. Rico Caldas}
\teacher{Julien Lecoeur}
\semester{Autumn 2015}
\cours{Mobile Robots}
\title{Miniproject}
\subtitle{Altitude control \\ \small{Project organization}}



\begin{document}
\thispagestyle{empty}
\maketitle

\section{Sensors}
\subsection{Sonar}
The sonar delivers a high resolution and low noise measurement compared to the other sensors (\SI{1}{\centi\meter} resolution with noise in the same order of magnitude).
But it's a measure of height (as opposed to a measure of altitude) and is not sufficent to control the altitude.
Furthermore, its sampling frequency (\SI{40}{\hertz}) is lower than the frequency of our \emph{control-loop}.

\subsection{Barometer}
The barometer is less precise than the sonar (typical \emph{RMS noise} of \SI{25}{\centi\meter} and a resolution of \SI{20}{\centi\meter}).
It returns an absolute altitude measurement, but it's drift is very limiting.
Also the changes of pressure induced by the environment (e.g.\ closing or opening a door) will distort the measurements.
The sampling frequency is the same as the sonar's when using the ultra high resolution mode (\SI{40}{\hertz}).

\subsection{Accelerometer}
The accelerometer isn't an altitude sensor, but provides us with information about the quadrotor movements.
Its bias is very important and the measurement can't just be integrated twice to yeild a position information without being fused with other sensors.
Its sampling frenquency is \SI{250}{\hertz} and thus the only one coming close to our \emph{control-loop} frequency. 

\section{Altitude determination}
Giving the fact that none of the sensors individually meets all of our needs (particularly the box that is inserted as a disturbance), we need to fuse two or more of them. 

\subsection{Accelerometer \& sonar}
By applying a \emph{low-pass} filter on the sonar and adding the double integral of a \emph{high-passed} accelerometer signal, we have low-noise reference to the ground while also measuring actual movements of the quad (as opposed to changes of the environment) with a high frequency.
The limitation of this setup is, that setpoint changes can only be realtive to the current altitude or height and never absolute.

\subsection{Barometer \& sonar}
The frequency at which this setup could estimate the altitude is probably too slow for it to be used for control.
Also, the barometer's drift is a big problem.
A possibility would be to use the barometer for the high frequency changes and the sonar for the steady-state.
This could yeild similar results as the accelerometer \& sonar setup.

\subsection{Accelerometer, barometer \& sonar}
A quite complex \emph{Kalman filter} which tracks altitude, height, vertical speed, and barometer as well as accelerometer bias could handle the fusion of all three sensors.

\subsection{Solution kept}
In order to follow an agile developing process and have a minimal working version as soon as possible, we will implement a complementary filter with the accelerometer and sonar.
A possible iteration would be to add the barometer.

\section{Control}
We plan to use two cascaded \emph{PIDs} to control the quadrotor's velocity and position. Like that we can split the problem by tuning two controllers separately.

\section{Planning}
\begin{table}[H]
	\centering
	\caption{Planning}
	\label{tab:planning}
	\begin{tabular}{l|l|l}
        \hline
        \textbf{Task} & Timespan & Description \\
        \hline
		\textbf{Altitude estimation} & 23.11 - 29.11 (2 students) & Design and tuning of the complementary filter \\
		\textbf{Altitude control}    & 30.11 - 06.12 (2 students) & Design and tuning of the 2 cascaded PID       \\
		\textbf{Tests}               & 07.12 - 13.12 (2 students) & Test flights, final tunings                   \\
        \hline
	\end{tabular}
\end{table}

\end{document}
